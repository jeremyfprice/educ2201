\documentclass[12pt]{article}
\usepackage[margin=0.8in]{geometry}
\usepackage{fontawesome}
\usepackage{fontspec}
\usepackage[hidelinks]{hyperref}
\usepackage{pdfpages}
\usepackage{csquotes}
\MakeOuterQuote{"}
\setmainfont{Palatino Linotype}
\setsansfont{Myriad Pro}
\setmonofont{Andale Mono}
\usepackage{microtype}

\usepackage{graphicx}
\setkeys{Gin}{width=\linewidth,totalheight=\textheight,keepaspectratio}
\graphicspath{{glogster-img/}}


\setlength{\parindent}{0em}
\setlength{\parskip}{12pt}

\definecolor{fsuMaroon}{RGB}{155,0,33}
\definecolor{googleBlue}{RGB}{68,134,248}
\definecolor{googleBlue2}{RGB}{53,122,232}
\definecolor{sbGray}{HTML}{F6F6FF}
\definecolor{googleRed}{HTML}{D0402F}
\definecolor{glogsterGreen}{HTML}{0BB04A}

\newcommand{\dueDate}[1]{\textbf{\textcolor{fsuMaroon}{#1}}}

\newcommand{\stepNo}[1]{\noindent\hangindent2em\textit{Step~#1.}}

\newcommand{\figImg}[1]{\begin{center}\frame{\includegraphics[width=0.35\linewidth]{#1.png}}\end{center}}

\newcommand{\buttonText}[1]{\fcolorbox{googleBlue2}{googleBlue}{\color{white}\textsf{\textbf{#1}}}}

\newcommand{\buttonTextRed}[1]{\fcolorbox{googleRed}{googleRed}{\color{white}\textsf{\textbf{\uppercase{#1}}}}}

\newcommand{\buttonTextGreen}[1]{\fcolorbox{glogsterGreen}{glogsterGreen}{\color{white}\textsf{\textbf{\uppercase{#1}}}}}

\newcommand{\tabText}[1]{\fcolorbox{gray}{sbGray}{\color{black}\textsf{#1}}}

\newcommand{\linkText}[1]{\textcolor{googleBlue}{\textsf{\textbf{#1}}}}

\newcommand{\keyText}[1]{\textsf{\faKeyboard~#1}}

\begin{document}

\begin{center}
\noindent\large\dueDate{\faBook}~~\textsf{\textit{Reference Sheet:} Glogster}\normalsize\\
\rule{4in}{1pt}
\end{center}

This reference sheet will lead you through the process of creating a Glog (interactive multimedia posters that can also be printed out) using the online tool Glogster.

\stepNo{1} Visit \url{https://edu.glogster.com/}. It is important to add \url{edu} before {glogster.com}. Click on the \buttonTextGreen{Get Started} button to get started.
\figImg{1}

\stepNo{2} Click on the \buttonTextGreen{Educator} button to sign up for an educator account with Glogster.
\figImg{2}

%\newpage

\stepNo{3} Click the \buttonTextGreen{TRY} button under the heading "Try out a paid license with a 7 day free trial" in the lower right corner of the screen. Once the 7 days are up, you will not lose anything and your account will revert to a free, rather than paid, account.
\figImg{3}

\newpage

\stepNo{4} Enter your email address [1], a username that you will remember [2], and a password that you will remember [3] in the appropriate text boxes. When you are ready, click on the \buttonTextGreen{Start now} button.
\figImg{4}

\stepNo{5} Your account is now active! You will receive an email almost immediately alerting you of this fact. You will be brought to your Dashboard, where Glogster will ask for further information about you. Select your country of residence from the list, then your state, then your city, and then your school. It does not matter which Fairmont State University you select. If your school is not listed, choose \linkText{- Add School -} at the bottom of the list and type in the name of your school. When you are ready, click on the \buttonText{next step} button.
\figImg{5}

\stepNo{6} You may indicate your interest [1] and role [2] (if you are just a student and you don't work in a school, you can choose any of the role options). When you are ready, click on the \buttonText{Save} button.
\figImg{6}

%\newpage

\stepNo{7} The next step is to start working on your glog. Click on the \buttonTextGreen{Create New Glog} button towards the top of your dashboard.
\figImg{7}

\stepNo{8} You will be presented with an array of different templates immediately under the \buttonTextGreen{Create New Glog} button. \linkText{Vertical} and \linkText{Horizontal} are the most flexible and allow for the most customization. When you have found the one you want, click on it.
\figImg{8}

\stepNo{9} Once you click on the template, you will be taken to the Glog design screen, with the \texttt{Tools} options (Text, Graphics, Image, Wall, Page, Audio, Video, Data) displayed.
\figImg{9}

\stepNo{10} The first part of this reference sheet is to place an uploaded image on the Glog. Click on the \texttt{Image} option and then click on the upward facing arrow to upload an image. Choose an image from your computer using the file select box.
\figImg{10}

\newpage

\stepNo{11} Glogster will upload the image and then "process" it. Your image will then show up in the box. Now click on the picture you just uploaded.
\figImg{11}

\stepNo{12} Your picture will then show up again on the right side of the box. If you'd like to add a frame, click on the \tabText{Add Frame} button.
\figImg{12}

\stepNo{13} A new box will pop up displaying your options for frames. When you've found the frame you want to use, click on it and a preview will be displayed.
\figImg{13}

\newpage

\stepNo{14} Click on the \buttonTextGreen{Use It} button to use the image in its frame.
\figImg{14}


\stepNo{15} Click on the triangle to minimize the box. You can always get the tools back by clicking on the triangle to expand the menu again or the \buttonText{+} to display the tools.
\figImg{15}

\newpage

\stepNo{16} You will now see the image on your Glog. You can rotate it using the curved arrows or resize it using the straight arrows.
\figImg{16}

\stepNo{17} To set the background image, click on the \texttt{Wall} option. You can choose an image from the Gallery, a solid color, or upload an image by clicking on the upward facing arrow. Click on the image you want as the background and then click on \buttonTextGreen{Use It}. Your image will then be set as the background.
\figImg{17}

\stepNo{18} To add text to your Glog, click on the \texttt{Text} option. Choose the type of text you'd like on your Glog. Plain text is represented by the first two options in each category. You can explore the different categories. When you find the one you'd like, click on it and then minimize or close the options box.
\figImg{18}

\stepNo{19} Double-click on your new text bubble or box. A number of different options will appear. But first, type the text you want into the bubble or box. Then you can choose the typeface, size, and color you would like. If you click on the \tabText{AUTO size} box under the size options, the size will adjust as you type so that everything fits in your bubble or box.
\figImg{19}

\stepNo{20} Click off the bubble or box and options will appear just like with images. You can resize it with the straight arrows and rotate it with the curved arrows.
\figImg{20}

\stepNo{21} Delete other elements on the page by clicking on the element and then clicking on the trash can. It will disappear.
\figImg{21}

\stepNo{22} Glogster contains a large number of clip art items to choose from. Click on the \texttt{Graphics} option to start selecting them.
\figImg{22}

\stepNo{23} When you find the one you want, click on it. The graphic will appear on the right side of the box. Then click on the \buttonTextGreen{Use It} button.
\figImg{23}

\stepNo{24} You can rotate or resize it with the curved or straight arrows.
\figImg{24}

\stepNo{25} You can combine elements by adding some text to your graphics if you'd like.
\figImg{25}

\stepNo{26} Glogster autosaves your work. You do need, however, to save your work occasionally (and you do need to give it a name). Click on the \buttonTextGreen{SAVE} button near the top of the screen.
\figImg{26}

\stepNo{27} Type the name of your Glog in the box that appears. Another set of options will appear. If you are unsure, choose "No category" and "No Topic" from the list; otherwise, choose the appropriate category and topic. You may select one or more grade levels. For the moment, make sure that the \texttt{UNFINISHED} option is selected. When you are ready, click the \buttonTextGreen{SAVE} button. Glogster will then save your Glog.
\figImg{27}

\stepNo{28} You will then see a message saying that your glog has been saved.  Your options are to continue editing your glog by clicking \buttonTextGreen{BACK TO EDIT}, preview your glog by clicking \buttonTextGreen{Preview}, or return to your dashboard by clicking \buttonTextGreen{DASHBOARD}.
\figImg{28}

\stepNo{29} In order to submit a your glog to TaskStream, save your glog again and make sure that you select the \texttt{Public} option. When notified that your glog has been saved, choose the \buttonTextGreen{Preview} option.
\figImg{29}

\stepNo{30} While viewing the Preview, copy the web address of your glog. This is the link that you will share with your teams and with your professor.
\figImg{30}

\stepNo{31} \textbf{And that's it! Congratulations on creating your glog!}
\end{document}
