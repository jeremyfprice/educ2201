\documentclass{report}
\usepackage{csquotes}
\usepackage{fontspec}
\usepackage{fontawesome}
\usepackage{enumitem}
\setmainfont{Patrick Hand}%OFL Sorts Mill Goudy}
\setsansfont{Palatino Linotype}%Open Sans}
\usepackage[margin=1in,portrait]{geometry}
%\newcommand{\attrib}[1]{%
%\nopagebreak{\raggedleft\footnotesize #1\par}}

\newcommand{\QandA}[2]{
	\noindent\faDoubleAngleRight\textsf{~~#1}\\#2\\
	\bigskip
}

\begin{document}

\noindent \textsf{Thank you for completing the EDUC2201 Syllabus "Quiz." Here are some of the kinds of responses I am expecting from your reading of the EDUC2201 Syllabus.}

\bigskip

\QandA{How often should you bring your syllabus to class?}{You should bring your syllabus to class \underline{every day}~~\faSmile~~It contains important information and the answers to many questions can be found in the syllabus. This practice gets you in the habit of having materials on hand to help you succeed.}

\QandA{What are your three "jobs" for the course?}{Your three jobs for the course are:\\ \indent \faHandLeft~Develop an academic relationships with me as your professor (as you should with all your professors),\\ \indent \faHandLeft~Get to know and learn how to use your course materials, and \\ \indent \faHandLeft~Consider your long-term returns from putting effort into this course.}

\QandA{What do you need to do them well?}{Much of this depends on you. You might want to drop by my office and we can talk about your career goals. That way I can help connect the work we are doing in class with what you hope to do in the future. I would definitely recommend buying a folder or three-ring binder to hold your materials. Lastly, I would need some time set aside for thinking and some imagination to help me consider those long-term returns. If you are unclear about what \underline{YOU} would need, please come and see me during my Student Drop-In Hours, and we can discuss it.}

\QandA{When is the best time to send an email to the professor for an excused absence?}{The best time to send me an email is \textbf{BEFORE CLASS}. If you forget, you can still let me know up to 24 hours after class.}

\QandA{TRUE or FALSE: The Professor will notice patterns of absences, excused and unexcused, and ask you about it.}{\emph{TRUE!} I work very hard to get to know each student and figure out how to best support each student's success. This means I will be on the lookout for you in class and I will approach you if I notice a pattern of absences.}

\QandA{Why is class participation important?}{Class participation does a number of things for you, me, and your classmates. It helps you make important connections between your own experiences and the content of the course as well as to make connections between different parts of the course content. Your classmates may be able to make connections more effectively if they see \underline{you} making similar connections. It also helps me see what you are learning (making connections) and helps me understand how to better support you.}

\QandA{What does quality participation look like?}{Lots of different ways:\\ \indent \faHandLeft~Participating in pair-share activities, \\ \indent \faHandLeft~\underline{Asking questions}, \\ \indent \faHandLeft~Helping others, \\ \indent \faHandLeft~Taking the conversation to a new level, \\ \indent \faHandLeft~\underline{Taking risks}, \\ \indent \faHandLeft~Drawing on the readings or personal experiences.\\ \noindent There are other ways, and we can figure those out together.}

\QandA{How will your class notes be checked?}{Class notes are an important practice to develop and they help you take what you've learned from one class session to another. The way that I will check your notes is by \MakeUppercase{requiring you to include \underline{specific ideas from class} in your reflection papers}. This way, you are free to take notes the way that works best for you. However, keep in mind that you will be required to \underline{CITE} your ideas from particular class sessions. That means including the date that the idea was discussed.}

\QandA{Do you agree with Richard Curwin that education is about giving learners 2nd, 3rd, 4th chances? Why or why not?}{This is up to you, but if you don't agree I hope to help you change your mind by the end of the semester. Making mistakes is how we learn, and giving learners more than one chance is a way of helping them show you that they are learning.}

\QandA{What 2nd chance do you have with your assignments?}{You have 1 week to resubmit an assignment if you fell like that you have learned from my feedback on your work. You should seriously consider and include my feedback when you resubmit it.}

\QandA{Why are metacognitive activities like Surprise Tweets and Metacognitive Checks important for learning?}{These activities are important for two major reasons:\\\indent\faHandLeft~These practices help you reflect on what you have been learning to make it more "permanent," and \\ \indent \faHandLeft~I have a sense of what you are taking away from my class.\\Another reason, however, is that these help you stay in the schoolwork "zone," keeping you connected with the class, the materials, and your professor.}

\QandA{Which statement best describes your mindset towards this course?}{This really depends on \underline{\MakeUppercase{You}}, but I am very interested in working with you to develop a thoughtful approach to technology in teaching and learning.}

\end{document}