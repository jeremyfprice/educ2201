%!TEX program = lualatex

\documentclass[12pt]{article}
\usepackage[margin=1in]{geometry}
\usepackage{fontspec}
\usepackage{csquotes}
\usepackage{booktabs}
\usepackage{arydshln}
%\usepackage[usenames,dvipsnames]{xcolor}
\usepackage{color}
\MakeOuterQuote{"}
\setmainfont{Palatino Linotype}
\setsansfont{Architects Daughter}
\usepackage{microtype}

\definecolor{darkgray}{gray}{0.4}

\setlength{\parindent}{1em}
\setlength{\parskip}{12pt}

%\newcommand{argPrompt}[2]{\noindent\textbf{#1}: \textsf{#2\ldots}\\}

\begin{document}
\noindent\large\textbf{PROMPT: \textit{Obesity is a common health problem today because of\\technology}}\normalsize

It is certainly true that obesity and the illnesses that come along with obesity have become more and more common today than they were in the past. This is an issue of serious concern as treatment of these illnesses have both economic and emotional costs due to healthcare expenses, people spending less time at work and therefore not contributing as much to the local and national economy, and the stress and concern that loved ones feel for those who are ill.

I disagree, however, with the claim that technology is the main cause of obesity. There is a significant difference between correlation and causality. Just because access to and use of technology has increased along with the incidence of obesity, technology does not necessarily cause obesity. Instead, I believe that decreased access to healthy foods have had a greater impact on obesity than technology.

It is often said in statistics that "correlation does not equal causation," that just because two things appear to grow at the same time, one does not necessarily lead to the other. A number of statisticians have provided examples that show how you can't make this kind of assumption. The "Spurious Correlation" blog provides many humorous examples, such as the correlation between the margarine consumption rates in the US and the divorce rates in the state of Maine. Biologist Stephen Jay Gould has written a book where he shows very convincingly that IQ scores are not a good measure of a person's intelligence. Similarly, it is difficult to authoritatively state with certainty that technology is causing today's struggle with obesity. Instead, one can say at best that access to and use of technology has risen over time, and so has rates of obesity.

Another factor that merits examination is that many people have less access to healthy food than when there was a lower obesity rate. Fewer people grow their own food. Many areas of the nation, particularly poor rural and urban areas, are considered "food deserts," where the only place to buy food is a gas station or convenience store. Along these lines, there have been a number of scientists, chefs, and authors (particularly well known chef and personality Christopher Kimball of America's Test Kitchen) have found that food companies are purposefully injecting ingredients that lead to obesity because it leads people to want more, and therefore buy more.

In conclusion, obesity is certainly a real and important issue for our society to address. However, blaming technology will not adequately fix this problem. Instead, it is important to look very closely at our access to healthy foods.

\newpage

\noindent\large\textbf{ARGUMENT’S CLAIM:} \small\textsf{\textcolor{darkgray}{The claim is the “main point” of the argument\ldots}}\normalsize\\
\bigskip
\bigskip

\noindent\large\textbf{REASONS:} \small\textsf{\textcolor{darkgray}{Reasons support the claim and explain what the claim is about\ldots}}\normalsize\\
\bigskip
\bigskip

\noindent\large\textbf{EVIDENCE:} \small\textsf{\textcolor{darkgray}{Evidence provides the “backing” for the reasons\ldots}}\large\\

\begin{tabular}{|p{1in}|p{2.5in}|p{2.5in}|}
%\toprule
 & Clear and Evident & Need to Check\\
 \toprule
 \small\raggedright\textbf{Facts and Events} (from the speaker's perspective) & & \\[30ex]
 \midrule
 \small\raggedright\textbf{Thoughts and Feelings} & & \\[30ex]
 \bottomrule
\end{tabular}
\normalsize
\end{document}
