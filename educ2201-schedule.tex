%!TEX program = lualatex

\documentclass{tufte-handout}
%\documentclass{article}
\usepackage{hyperref}
\usepackage{amsmath}
\usepackage{titlesec}
\usepackage{framed}
\usepackage{todonotes}
\usepackage{arydshln}
\usepackage{fontawesome}
\usepackage{pdfpages}
\usepackage{csquotes}
\MakeOuterQuote{"}
\usepackage{newpxtext,newpxmath}
\usepackage{soul}
\useosf % old-style figures in text, not in math

\titleformat
{\part} % command
[display] % shape
{\bfseries\Large\itshape} % format
{Story No. \ \thechapter} % label
{0.5ex} % sep
{
    \rule{\textwidth}{1pt}
    \vspace{1ex}
    \centering
} % before-code
[
\vspace{-0.5ex}%
\rule{\textwidth}{0.3pt}
] % after-code

% Set up the images/graphics package
\usepackage{graphicx}
\setkeys{Gin}{width=\linewidth,totalheight=\textheight,keepaspectratio}
\graphicspath{{syllabus-img/}}

\title{Teaching with Technology}
\author{Dr. Jeremy Price}
\date{Spring 2015}  % if the \date{} command is left out, the current date will be used

% The following package makes prettier tables.  We're all about the bling!
\usepackage{booktabs}

% The units package provides nice, non-stacked fractions and better spacing
% for units.
\usepackage{units}

% The fancyvrb package lets us customize the formatting of verbatim
% environments.  We use a slightly smaller font.
\usepackage{fancyvrb}
\fvset{fontsize=\normalsize}

% Small sections of multiple columns
\usepackage{multicol}

% Provides paragraphs of dummy text
\usepackage{lipsum}

\usepackage{microtype}

\newcommand{\tabpq}{\faQuestionCircle\medspace\textit{Priming Questions}}
\newcommand{\tabread}{\faBook\medspace\textit{Readings}}
\newcommand{\tabperformance}{\faTasks\medspace\textit{Performances}}
\newcommand{\tabtools}{\faWrench\medspace\textit{Tools}}
\newcommand{\tabtweet}{\faLightbulbO\medspace\textit{Reflection Task} & Learning Tweet Due Friday \\}
\newcommand{\tabcheck}{\faLightbulbO\medspace\textit{Reflection Task} & Metacognitive Check Due Friday (no Learning Tweet) \\}

\newenvironment{tabsched}
	{\small
	\begin{tabular}{p{1.5in}p{4.5in}}
	\toprule}
	{\bottomrule
	\end{tabular}
	\normalsize\\}

\newenvironment{specweek}
	{\begin{center}
		\fontseries{b} \faBullhorn \medspace Special Week: }
		{\medspace \faBullhorn \fontseries{m}
	\end{center}}

\newcommand{\weekone}{August 17-21}
\newcommand{\weektwo}{August 24-28}
\newcommand{\weekthree}{August 31-September 4}
\newcommand{\weekfour}{September 7-11}
\newcommand{\weekfive}{September 14-18}
\newcommand{\weeksix}{September 21-25}
\newcommand{\weekseven}{September 28-October 2}
\newcommand{\weekeight}{October 5-9}
\newcommand{\weeknine}{October 12-16}
\newcommand{\weekten}{October 19-23}
\newcommand{\weekeleven}{October 26-30}
\newcommand{\weektwelve}{November 2-6}
\newcommand{\weekthirteen}{November 9-13}
\newcommand{\weekfourteen}{November 16-20}
\newcommand{\weekfifteen}{November 30-December 4}
\newcommand{\laborday}{Labor Day, Monday, September 7 (no class)}
\newcommand{\roshhashanah}{Rosh Hashanah, Monday, September 14 (class via VoiceThread)}
\newcommand{\yomkippur}{Yom Kippur, Wednesday, September 23 (class via VoiceThread)}
\newcommand{\midsemester}{Mid-Semester Point}
\newcommand{\acmhe}{Dr. Price at ACMHE Conference, Friday, October 9}
\newcommand{\thanksgiving}{Thanksgiving Recess, November 23-27 (no class)}
\newcommand{\finisemester}{Last Day of Class, Friday, December 4}

\newcommand{\listmon}{\item[Monday] \hfill \\}
\newcommand{\listwed}{\item[Wednesday] \hfill \\}
\newcommand{\listfri}{\item[Friday] \hfill \\}
\newenvironment{daywu}
	{\textbf{\underline{Warm-Up:}} \hfill \\
	\begin{itemize}}
	{\end{itemize}}
\newenvironment{dayact}
	{\textbf{\underline{Activities:}} \hfill \\
	\begin{itemize}}
	{\end{itemize}}
\newenvironment{dayref}
	{\textbf{\underline{Reflection:}} \hfill \\
	\begin{itemize}}
	{\end{itemize}}
\newenvironment{weeksched}
	{\noindent
	\begin{description}}
	{\end{description}
	\newpage}

% Set up the spacing using fontspec features
\renewcommand\allcapsspacing[1]{{\addfontfeature{LetterSpace=15}#1}}
\renewcommand\smallcapsspacing[1]{{\addfontfeature{LetterSpace=10}#1}}

\begin{document}
\begin{fullwidth}

\section{Unit 0: Getting Situated and Started}

\subsection{Week of \weekone}

\begin{tabsched}
	\tabpq & What are some of the rules of the "program" for "doing school"? \\
	& Should an effort be made to "deprogram" students from "doing school"? \\
	& What does the author mean by "surprise," and why is surprise important for learning? \\
	& Which components of a reflective classroom have you seen? \\
	\midrule
	\tabread & How Deprogramming Kids From How To Do School Could Improve Learning (\url{http://goo.gl/pnkkRM}) \\
	& Surprise Journal: Notice the Unexpected (\url{http://goo.gl/h2ESqJ}) \\
	& 8 Components of a Reflective Classroom (\url{http://bit.ly/1OXeCrp}) \\
	\midrule
	\tabtools & TaskStream (\url{https://www.taskstream.com/}) \\
	& Microsoft Word 365 \\
	\midrule
	\tabtweet
\end{tabsched}
\begin{weeksched}

\listmon
\begin{daywu}
	\item Name and one thing learned over the break
\end{daywu}
\begin{dayact}
	\item SHORT intro, skip reading syllabus out loud
	\item Do drawing activity: what does it look like to teach with technology? Short discussion
	\item I will read poem out loud, ask students to contribute to Google Doc (anonymously) with initial reactions to each poem, discuss 10am: \url{http://goo.gl/FTNFcm}, 12pm: \url{http://goo.gl/L8mRBg}
\end{dayact}
\begin{dayref}
	\item Use reflection sentence stem using Google Doc "What is something that surprised you in class today?" 10am: \url{http://goo.gl/VKflFk}, 12pm: \url{http://goo.gl/SZWHql}
\end{dayref}

\listwed
\begin{daywu}
	\item Name and something you are excited about for the coming semester
\end{daywu}
\begin{dayact}
	\item Pause for questions about the course, requirements, or due dates
	\item Watch video in three sections (\url{http://goo.gl/CqgQxW})
	\begin{itemize}
		\item What do you notice? What do you find interesting, what do you recognize, what good stuff is going on, what not so good stuff is going on?
		\item Write responses into Word Doc (on Blackboard)
		\item Short discussion
		\begin{itemize}
			\item Make connections to readings, especially Deprogramming article
			\item What does "doing school" look like for you?
		\end{itemize}
	\end{itemize}
\end{dayact}
\begin{dayref}
	\item Complete Passport Action Plans
\end{dayref}

\listfri
\begin{daywu}
	\item Name and an event that has had a big impact on your life as a student
\end{daywu}
\begin{dayact}
	\item Present Reflection Framework
	\item Watch video (\url{http://bit.ly/1N7pK7N})
	\begin{itemize}
		\item What do you notice? What are the teachers discussing in terms of what happened? What are the teachers discussing in terms of what to do next?
		\item Running responses in TodaysMeet (on Blackboard)
		\item Short discussion
		\begin{itemize}
			\item Make connections to readings, especially reflection
			\item Where do you see the connections with the reflection framework?
			\item Where do you see connections with the attributes of a reflective classroom? (esp., mutual respect, culture of questioning, diverse viewpoints)
		\end{itemize}
	\end{itemize}
\end{dayact}
\begin{dayref}
	\item Have students complete Passport Action Plan, make sure to mention about prizes and connect with the idea of reflection
\end{dayref}
\end{weeksched}

\section{Unit 1: Communicating the Role of Technology in Education}

\subsection{Week of \weektwo}

\begin{tabsched}
	\tabpq & According to Grant Wiggins, what are the differences between an \emph{argument} and a persuasive essay? \\
	& According to John Spencer, how is it \emph{about} the technology? \\
	\midrule
	\tabread & Argument\textemdash{}the Core of the Common Core\textemdash{}and a clarifying example (\url{http://goo.gl/1nhpHs}) \\
	& Actually, It Is About the Technology (\url{http://goo.gl/BE79FJ}) \\
	\midrule
	\tabtools & Padlet (\url{https://padlet.com/}) \\
	& ScreenChomp (\url{http://www.techsmith.com/screenchomp.html}) \\
	\midrule
	\tabcheck
	\midrule
	\tabperformance & Syllabus "Quiz" Due Monday, August 24 \\
\end{tabsched}
\begin{weeksched}

\listmon
\begin{daywu}
	\item Silicon connections: \textbf{\emph{\hl{REMEMBER TO BRING SILICON CHUNK TO CLASS.}}}
\end{daywu}
\begin{dayact}
	\item Re-Present Reflection Framework
	\item Watch video (\url{http://bit.ly/1N7pK7N})
	\begin{itemize}
		\item What do you notice? What are the teachers discussing in terms of what happened? What are the teachers discussing in terms of what to do next?
		\item Running responses in TodaysMeet (on Blackboard)
		\item Short discussion
		\begin{itemize}
			\item Make connections to readings, especially reflection
			\item Where do you see the connections with the reflection framework?
			\item Where do you see connections with the attributes of a reflective classroom? (esp., mutual respect, culture of questioning, diverse viewpoints)
		\end{itemize}
	\end{itemize}
	\item Have teams write out themes that emerge from TodaysMeet, and then share with class
\end{dayact}
\begin{dayref}
	\item Use reflection sentence stem using Google Doc "What is something that surprised you in class today?" 10am: \url{http://goo.gl/VKflFk}, 12pm: \url{http://goo.gl/SZWHql}
\end{dayref}

\listwed
\begin{daywu}
	\item Brad Paisley video: Ask for "What is the message or messages about technology in Brad Paisley's music video?" Think-Pair-Share: \url{http://goo.gl/fvQqzA}
\end{daywu}
\begin{dayact}
	\item Make \hl{transition to first unit}, which will help us think about what technology in learning and teaching is about and also to help prepare for the Core exam. Students will be writing an argument, which is one of the writing tasks on the writing section of the Core exam.
		\item What's an argument? Create a list on the board.
		\item Identify phrases that Grant Wiggins would say fits for argument and what fits for "persuasive essay." Argument is about demonstrating an understanding.
		\item Take photo of board once it is all done.
\end{dayact}
\begin{dayref}
		\item \textbf{\hl{LEAVE 15 MINUTES FOR SURPRISE TWEET:}} Ask them to complete the Surprise Tweet and then ask for some people to share.
		\item Remember that Syllabus "Quiz" is due by end of day
\end{dayref}

\listfri
\begin{daywu}
	\item Share the name of one technology that has made a difference in your life
\end{daywu}
\begin{dayact}
	\item Share example from me and kids (Adventures in Fugawiland, \url{http://goo.gl/vEVmVK})
	\item ScreenChomp, Most Memorable Experience Learning with Technology
	\item Ask students to share what they talked about and then what they noticed: were experiences surprising, similar, different?
\end{dayact}
\begin{dayref}
	\item Sentence stem for reflection: \hl{My latest thinking about technology in learning and teaching is\ldots}
\end{dayref}
\end{weeksched}

\subsection{Week of \weekthree}

\begin{tabsched}
	\tabpq & According to Tom Whitby, what are the differences between \textit{then} and \textit{now}? \\
	& What is the advice Bill Ferriter gives to help make students more engaged in learning with technology? \\
	\midrule
	\tabread & The Longer View: EdTech and 21st-Century Education (\url{http://goo.gl/iuJk59}) \\
	& Are Kids Really Motivated by Technology? (\url{http://goo.gl/s3orWi}) \\
	\midrule
	\tabtweet
\end{tabsched}
\begin{weeksched}
\listmon
	\begin{daywu}
		\item Serial Storytelling: Once upon a time, there was a class full of college students\ldots
	\end{daywu}
	\begin{dayact}
		\item Review What's an Argument, and show students their word associations paying special attention to how Wiggins defined argument
		\item Go over the model argument, identify claim, evidence, reasons, diagram on whiteboard
		\item If time, use Reporter's Notebook to evaluate evidence
	\end{dayact}
	\begin{dayref}
		\item Choose one or both sentence stems: \hl{One thing I learned about arguments is\ldots One question I still have about arguments is\ldots{}}
	\end{dayref}

\listwed
	\begin{daywu}
		\item What is your favorite food?
	\end{daywu}
	\begin{dayact}
		\item Joint creation of Pepperoni Roll Rubric
		\item Review the rubric for the essay
	\end{dayact}
	\begin{dayref}
		\item Thumbs Up/Thumbs Down about \hl{rubrics and how they are used}; Thumbs Up/Thumbs Down about the \hl{argumentation essay}
	\end{dayref}

\listfri
	\begin{daywu}
		\item The best thing that happened all week was\ldots
	\end{daywu}
	\begin{dayact}
		\item Hyperbole video (\url{http://goo.gl/MUtnU4})
		\item APA citations
		\item Plickers
	\end{dayact}
	\begin{dayref}
		\item The first step I'm going to take to write my essay is\ldots
	\end{dayref}
\end{weeksched}

\section{Unit 2: Creating an Online Book for All Learners}

\subsection{Week of \weekfour}

\begin{specweek}\laborday\end{specweek}

\begin{tabsched}
	\tabpq & What is the ultimate goal of Universal Design for Learning? \\
	& What are the different \enquote{modes} according to UDL? \\
	& What examples of these different modes have you seen in your own educational career? \\
	\midrule
	\tabread & What Is Universal Design for Learning? (\url{http://goo.gl/V9YHNw}) \\
	& UDL At A Glance Video (\url{http://goo.gl/1xhgQh}) \\
	& UDL Questions \& Answers (\url{http://goo.gl/watsXV}) \\
	\midrule
	\tabtweet
	\midrule
	\tabtools & Socrative (\url{http://b.socrative.com/login/student/}) \\
	\midrule
	\tabperformance & Feedback from Writing Center Completed by Friday, September 11 \\
\end{tabsched}
\newpage
\subsection{Week of \weekfive}

\begin{specweek}\roshhashanah\end{specweek}

\begin{tabsched}
	\tabpq & According to Carol Dweck, why are the two mindsets\textemdash{}fixed and growth\textemdash{}important? \\
	& What are times that \emph{you} have taken on a fixed mindset? A growth mindset? \\
	\midrule
	\tabread & Mindsets: How to Motivate Students (And Yourself) (\url{http://goo.gl/N997iN}) \\
	& The Power of Believing You Can Improve (\url{http://goo.gl/ADvfw4}) \\
	\midrule
	\tabtools & Storyboard That! (\url{https://www.storyboardthat.com/}) \\
	& UDL BookBuilder (\url{http://bookbuilder.cast.org}) \\
	\midrule
	\tabtweet
	\midrule
	\tabperformance & Written Argument Due by Friday, September 18 \\
\end{tabsched}

\subsection{Week of \weeksix}

\begin{specweek}\yomkippur\end{specweek}

\begin{tabsched}
	\tabpq & What are the four Reciprocal Teaching Strategies? \\
	& What do the Reciprocal Teaching Strategies do for learners? \\
	\midrule
	\tabread & Reciprocal Teaching Strategies (\url{http://goo.gl/pIk8XA}) \\
	\midrule
	\tabtweet
	\midrule
	\tabperformance & UDL Book Outline Intermediate Performance Due by Monday, September 23 \\
	& UDL Book Storyboard Intermediate Performance Due by Friday, September 25 \\

\end{tabsched}

\subsection{Week of \weekseven}

\begin{tabsched}
	\tabpq & According to the authors, what are the benefits of students becoming digital authors? \\
	& According to the authors, why is it important for students to use technology in school? \\
	& What does it mean to write purposefully? \\
	\midrule
	\tabread & Creating Digital Authors (\url{http://goo.gl/nTCziC}) \\
	\midrule
	\tabtweet
\end{tabsched}

\subsection{Week of \weekeight}

\begin{specweek}\acmhe\end{specweek}

\begin{tabsched}
	\multicolumn{2}{c}{\faLaptop\medspace\textbf{Focus and Project Time}\medspace\faLaptop} \\
	\midrule
	\tabcheck
\end{tabsched}

\begin{specweek}\midsemester\end{specweek}

\section{Unit 3: Exploring the Many Sides of Diversity}

\subsection{Week of \weeknine}

\begin{tabsched}
	\tabpq & What is a WebQuest? \\
	& What can a student learn from engaging in a WebQuest? \\
	\midrule
	\tabread & \textit{Exploring the Many Sides of Diversity} WebQuest (\url{http://bit.ly/webquest-diversity})\\
	&  What is a WebQuest? (\url{http://goo.gl/M9HDsK}) \\
	& What are the Essential Parts of a WebQuest? (\url{http://goo.gl/737kms}) \\
	& What Kinds of Topics Lend Themselves to WebQuests? (\url{http://goo.gl/ZrD7r4}) \\
	\midrule
	\tabtools & See the \textit{Exploring the Many Sides of Diversity} WebQuest (\url{http://bit.ly/webquest-diversity}) \\
	\midrule
	\tabtweet
	\midrule
	\tabperformance & UDL Book Major Performance Due by Friday, October 16 \\
\end{tabsched}

\subsection{Week of \weekten}

\begin{tabsched}
	\tabtools & See the \textit{Exploring the Many Sides of Diversity} WebQuest (\url{http://goo.gl/sUcRU1}) \\
	\midrule
	\tabtweet
	\midrule
	\tabperformance & Diversity Poster PDF for Printing Due by Friday, October 23 \\
	& Diversity Poster Culminating Performance Due by Friday, October 30 \\
\end{tabsched}

\section{Unit 4: The Quest to Build a WebQuest}

\subsection{Week of \weekeleven}

\begin{tabsched}
	\tabpq & According to the authors, what does "understanding" mean? \\
	& How is this definition similar to and different from your own definition of understanding? \\
	& What do Generative Topics and Understanding Goals add to the teacher planning process? \\
	\midrule
	\tabread & Introducing Teaching for Understanding (\url{http://goo.gl/2F1CxV}) \\
	& Generative Topics (\url{http://goo.gl/slJbOc}) \\
	& What are Understanding Goals? (\url{http://goo.gl/qapL2r}) \\
	\midrule
	\tabtools & Prezi (\url{https://www.prezi.com/}) \\
	\midrule
	\tabtweet
\end{tabsched}

\subsection{Week of \weektwelve}

\begin{tabsched}
	\tabpq & How do Performances of Understanding help the learning process? \\
	& What are the different ways a teacher can use scaffolding in a WebQuest? \\
	\midrule
	\tabread & What are Performances of Understanding (\url{http://goo.gl/Om1hT1}) \\
	& Using the "Zone" to Help Reach Every Learner (\url{http://goo.gl/KCigdq}) \\
	& 24 Assessments That Don't Suck (\url{http://goo.gl/Qzu8Hv}) \\
	\midrule
	\tabtools & Google Sites (\url{https://sites.google.com/}) \\
	\midrule
	\tabtweet
\end{tabsched}

\subsection{Week of \weekthirteen}

\begin{tabsched}
	\multicolumn{2}{c}{\faLaptop\medspace\textbf{Focus and Project Time}\medspace\faLaptop} \\
	\midrule
	\tabtweet
	\midrule
	\tabperformance & Prezi Teaching for Understanding Map Intermediate Performance Due Friday November 13 \\
\end{tabsched}

\subsection{Week of \weekfourteen}

\begin{tabsched}
	\multicolumn{2}{c}{\faLaptop\medspace\textbf{Focus and Project Time}\medspace\faLaptop} \\
	\midrule
	\tabtweet
\end{tabsched}

\begin{specweek}\thanksgiving\end{specweek}

\subsection{Week of \weekfifteen}

\begin{tabsched}
	\tabpq & What are some of the "pictures of teaching" presented in these articles? \\
	& What experiences have you had as a learner that look like these "pictures of teaching"? \\
	\midrule
	\tabread & Mastering the Teaching Game (\url{http://goo.gl/VqVZVx}) \\
	& \#Teachingis Adapting (\url{http://goo.gl/lNUWTh}) \\
	& The Importance of Saying "I'm Sorry" (\url{http://goo.gl/gNKcBa}) \\
	\midrule
	\tabcheck
	\midrule
	\tabperformance & WebQuest Design Culminating Performance Due by Friday, December 4 \\
\end{tabsched}

\begin{specweek}\finisemester\end{specweek}

\end{fullwidth}
\end{document}